% Options for packages loaded elsewhere
\PassOptionsToPackage{unicode}{hyperref}
\PassOptionsToPackage{hyphens}{url}
%
\documentclass[
]{article}
\usepackage{amsmath,amssymb}
\usepackage{lmodern}
\usepackage{ifxetex,ifluatex}
\ifnum 0\ifxetex 1\fi\ifluatex 1\fi=0 % if pdftex
  \usepackage[T1]{fontenc}
  \usepackage[utf8]{inputenc}
  \usepackage{textcomp} % provide euro and other symbols
\else % if luatex or xetex
  \usepackage{unicode-math}
  \defaultfontfeatures{Scale=MatchLowercase}
  \defaultfontfeatures[\rmfamily]{Ligatures=TeX,Scale=1}
\fi
% Use upquote if available, for straight quotes in verbatim environments
\IfFileExists{upquote.sty}{\usepackage{upquote}}{}
\IfFileExists{microtype.sty}{% use microtype if available
  \usepackage[]{microtype}
  \UseMicrotypeSet[protrusion]{basicmath} % disable protrusion for tt fonts
}{}
\makeatletter
\@ifundefined{KOMAClassName}{% if non-KOMA class
  \IfFileExists{parskip.sty}{%
    \usepackage{parskip}
  }{% else
    \setlength{\parindent}{0pt}
    \setlength{\parskip}{6pt plus 2pt minus 1pt}}
}{% if KOMA class
  \KOMAoptions{parskip=half}}
\makeatother
\usepackage{xcolor}
\IfFileExists{xurl.sty}{\usepackage{xurl}}{} % add URL line breaks if available
\IfFileExists{bookmark.sty}{\usepackage{bookmark}}{\usepackage{hyperref}}
\hypersetup{
  hidelinks,
  pdfcreator={LaTeX via pandoc}}
\urlstyle{same} % disable monospaced font for URLs
\usepackage[margin=1in]{geometry}
\usepackage{graphicx}
\makeatletter
\def\maxwidth{\ifdim\Gin@nat@width>\linewidth\linewidth\else\Gin@nat@width\fi}
\def\maxheight{\ifdim\Gin@nat@height>\textheight\textheight\else\Gin@nat@height\fi}
\makeatother
% Scale images if necessary, so that they will not overflow the page
% margins by default, and it is still possible to overwrite the defaults
% using explicit options in \includegraphics[width, height, ...]{}
\setkeys{Gin}{width=\maxwidth,height=\maxheight,keepaspectratio}
% Set default figure placement to htbp
\makeatletter
\def\fps@figure{htbp}
\makeatother
\setlength{\emergencystretch}{3em} % prevent overfull lines
\providecommand{\tightlist}{%
  \setlength{\itemsep}{0pt}\setlength{\parskip}{0pt}}
\setcounter{secnumdepth}{-\maxdimen} % remove section numbering
\ifluatex
  \usepackage{selnolig}  % disable illegal ligatures
\fi

\author{}
\date{\vspace{-2.5em}}

\begin{document}

\hypertarget{codeclan---phs-rshiny-dashboard-group-project}{%
\section{CodeClan - PHS RShiny Dashboard Group
Project}\label{codeclan---phs-rshiny-dashboard-group-project}}

\hypertarget{april-2022}{%
\subsection{April 2022}\label{april-2022}}

\begin{center}\rule{0.5\linewidth}{0.5pt}\end{center}

\hypertarget{project-description-outline}{%
\subsection{Project Description
Outline}\label{project-description-outline}}

\hypertarget{group-members}{%
\subsubsection{Group Members:}\label{group-members}}

Colin Scotland, Mahri Stewart, Kahlen Cheung, Jerry Balloch

\hypertarget{roles-responsibilities-of-each-member}{%
\subsubsection{Roles \& responsibilities of each
member}\label{roles-responsibilities-of-each-member}}

\emph{Colin} Colin worked on viewing the percentage of beds occupied by
acute care patients across the 14 Health Boards in Scotland both before
and during COVID times. Statistical analysis of this data was also
calculated using a two independent means test with permutation to
generate a null distribution. He also drew up a dashboard wire frame,
worked on creating the dashboard itself, and regularly updated Trello.

\emph{Mahri} Mahri worked on demographic data both before and during
COVID times. Whilst considering Scotland as a whole, the differences in
total monthly/ quarterly admissions/ stays for acute care patients was
observed between: * age groups * sex (Male/Female) * and SIMD quintiles
(Scottish Index of Multiple Deprivation quintiles: 1(Most Deprived) -
5(Least Deprived)). Statistical analysis of the three were also
calculated for both pre-COVID and during COVID times. Creating and
updating a Trello project board with to-do tasks and scheduled meetings,
and working on the presentation and README file.

\emph{Kahlen} Kahlen worked on A\&E admissions both before and during
COVID times across Scotland as a whole. She also considered the
differences in waiting times (4 hours, 8+ hours, and 12+ hours) in A\&E
across Scotland and the same time frame. Statistical analysis on the
differences between the mean of total A\&E admissions \emph{before}
COVID, and the mean of total A\&E admissions \emph{after} COVID of this
data was also calculated.

\emph{Jerry} Jerry worked on the dashboard skeleton, appearance, and
input of relevant data and widgets. Creating a colour palette from the
PHS logo and Scotland as a whole, helped in keeping the overall
aesthetics of individual visualisations uniform. Geographical
information for total COVID cases was shown on maps created using the
leaflet() function in RShiny. It involved merging shapefile data,
containing the Polygon data of the different health board regions, with
another data set containing COVID case counts across each region. The
data was able to be joined using the health board reference id across
both data sets.

\emph{Everyone} Everyone worked together with regards to coming up with
a direction for this project. Though we had our individual roles, we
regularly presented what we had each achieved and our next steps were
decided upon in group discussions. Combining the individual analyses
into the dashboard so that it had a uniform aesthetic and an outline for
our presentation was also decided upon as a group. Group meetings were
attended by all and Trello was updated regularly with how everyone was
getting on/ our to do list.

\begin{center}\rule{0.5\linewidth}{0.5pt}\end{center}

\hypertarget{brief-description-of-dashboard-topic}{%
\subsubsection{Brief description of dashboard
topic}\label{brief-description-of-dashboard-topic}}

We were interested in considering whether there is a ``Winter Crisis''
as portrayed by the media by comparing data from before the pandemic
(2016-2019) with the data for COVID times (2020 to 2021).

Our dashboard contains information on bed occupancy across Scottish
Health Boards, demographics (age, sex, and deprivation scores), and A\&E
admissions and waiting times from Summer 2016 to Summer 2021. As a user,
you are able to change the dates (yearly quarters) so as to zoom in on a
time period of your choice. A map of Scotland's Health Boards is also
available with a note of the total COVID cases to date for each.

\hypertarget{stages-of-the-project}{%
\subsubsection{Stages of the project}\label{stages-of-the-project}}

\begin{itemize}
\tightlist
\item
  Taking time to individually view available datasets and consider
  routes of investigation
\item
  Choosing datasets
\item
  Planning \& dashboard wireframe
\item
  Cleaning and analysis of data with regards to our interests
\item
  Statistical investigation
\item
  Geographical inputs on map
\item
  Git branching \& version control
\item
  Construction of dashboard skeleton
\item
  Combining of individual work on dashboard
\item
  Testing dashboard
\item
  Presentation slides drawn up
\item
  Assigning presentation topics
\end{itemize}

\hypertarget{which-tools-were-used-in-the-project}{%
\subsubsection{Which tools were used in the
project}\label{which-tools-were-used-in-the-project}}

\begin{itemize}
\tightlist
\item
  Zoom - initial and weekend stand-ups
\item
  Trello - planning \& task allocation
\item
  Git/GitHub - collaboration \& version control
\item
  Rstudio - cleaning and analysis of data
\item
  Excalidraw - creating and editing dashboard wirefame
\item
  Slack for communication
\end{itemize}

\hypertarget{how-did-you-gather-and-synthesise-requirements-for-the-project}{%
\subsubsection{How did you gather and synthesise requirements for the
project?}\label{how-did-you-gather-and-synthesise-requirements-for-the-project}}

Investigation of open data provided by Public Health Scotland was
undertaken by each group member individually with the project questions
in mind. Group discussions were had with regards to which data sets
could be condsidered and after cleaning and analysis, the project was
brought together with clean data that provided a story to answer our
question. We prioritised having a working process. Whilst data was being
cleaned and analysed, the dashboard skeleton was being created so that
everything could be brought together quickly and efficiently.

\hypertarget{motivations-for-using-the-data-you-have-chosen}{%
\subsubsection{Motivations for using the data you have
chosen}\label{motivations-for-using-the-data-you-have-chosen}}

We used the dataset on bed occupancy to answer the question of how
genuine the so-called ``winter crisis'' is and because it was a simple
and clear metric with which to compare the demand on hospital beds
depending on time of year. The bed occupancy was also given as a
\emph{percentage} of total available beds which meant that it was
already normalised and comparable between different health boards
regardless of their total overall bed count. \\
The dataset \texttt{A\&E\ attendances\ and\ performance\ data} has
recorded the waiting time of patients being discharged to the health
board from the year of 2007 and 2022. It reviews the number of
attendances of each separated time range: the standard 4hrs, more than
8hrs and more than 12 hrs. We are able to discover the trend and
performance of A\&E waiting time, which allow us to answer the question-
``if winter crisis exists'', by comparing the performance differences of
the pre-COVID and post-COVID period.

\hypertarget{data-quality-and-potential-bias-including-a-brief-summary-of-data-cleaning-and-transformations}{%
\subsubsection{Data quality and potential bias, including a brief
summary of data cleaning and
transformations}\label{data-quality-and-potential-bias-including-a-brief-summary-of-data-cleaning-and-transformations}}

According to the About tab on the dataset page/dedicated page online,
the data hosted on the Scottish Health and Social Care Open Data
platform follows the open data standards set out by Public Health
Scotland, ensuring consistency across the platform.
(\url{https://www.opendata.nhs.scot/about}) The dataset may not be
biased because all NHS Scotland organisations can use this platform to
publish and share their open datasets that meet the requirements set out
in the Scottish Government Open Data Strategy (2015). Further, feedback
is requested from users as to which data they wish to see in the future
and how their experiences of using the available data, and before data
is released on the open data platform it is first assessed for
statistical disclosure.

To clean the datasets we;

\begin{itemize}
\item
  Determined as a group what questions we wanted to answer
\item
  Worked individually on separate datasets to remove unnecessary data
  and to transform the remaining data into relevant, usable objects that
  could provide an insight into the questions we were asking.
  Specifically;

  \begin{itemize}
  \tightlist
  \item
    Dates and timelines were adjusted between months/quarters/seasons as
    appropriate to facilitate comparisons between different datasets.
  \item
    Common variables were used to allow data to be joined (e.g.~health
    board codes, health board names, etc) to give a better overall idea
    of what was happening in the bigger picture.
  \end{itemize}
\end{itemize}

Cleaning and transformation was all done in RStudio, predominantly using
the \texttt{tidyverse} and \texttt{janitor} packages.

\hypertarget{how-is-the-data-stored-and-structured}{%
\subsubsection{How is the data stored and
structured}\label{how-is-the-data-stored-and-structured}}

{[}Hint: This page offers a good starting point for understanding the
data structure:
\url{https://guides.statistics.gov.scot/article/34-understanding-the-data-structure}{]}

The data is in the form of cleaned .csv files saved into a
\texttt{clean\_data} folder within the Github repository. Each data set
contains a time reference of some kind (month or quarter) or a reference
to a health board.

This means that the data can be linked by month/quarter or by health
board information to allow for further analysis and interpretation.

Benefits of storing the data like this are;

\begin{itemize}
\tightlist
\item
  The data is already clean
\item
  Further analysis can be performed simply by reading in the .csv
\item
  By maintaining date and location data the datasets can be added to in
  future to look for developing trends in different locations over
  longer periods.
\end{itemize}

\hypertarget{ethical-and-legal-considerations-of-the-data}{%
\subsubsection{Ethical and legal considerations of the
data}\label{ethical-and-legal-considerations-of-the-data}}

There are no ethical concerns, because the datasets are covered by the
Open Government License (provided by Public Health Scotland), which
means you are encouraged to use and adapt, combine, explore, distribute,
and publish in your own project the information that is available under
this licence freely and flexibly. There are no personal information or
identifiers included in the datasets. You must, however, acknowledge the
source of the Information in your product or application by including or
linking to any attribution statement specified by the Information
Provider(s) and, where possible, provide a link to this licence:
\url{https://www.nationalarchives.gov.uk/doc/open-government-licence/version/3/}

\end{document}
